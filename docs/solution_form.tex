\documentclass[a4paper,12pt]{article}
\usepackage{polski}
\usepackage[utf8]{inputenc}

\usepackage{amsmath}
\usepackage{amsfonts}
\usepackage{bm}

\DeclareMathOperator*{\argmin}{argmin}

\title{Problem KtoMiPizzęZżera}
\date{}

\begin{document}

\maketitle

Dane:
$$
    n \text{ -- liczba potrzeb}
$$
$$
    m \text{ -- liczba pizz}
$$
$$
    k \text{ -- liczba składników}
$$
$$
    p \text{ -- liczba kawałków w jednej pizzy}
$$
$$
    L_i \text{ -- zbiór składników lubianych dla potrzeby $i$-tej}
$$
$$
    N_i \text{ -- zbiór składników nielubianych dla potrzeby $i$-tej }
$$
$$
    S_i \text{ -- zbiór składników pizzy $i$-tej }
$$
$$
    \pmb{c}_{m \times 1} \text{ -- wektor cen pizz}
$$

$$
    \left[ \pmb{A}_{m \times k} \right]_{i,j}
    = \begin{cases}
        1 & \text{if $s_j \in S_i$} \\
        0 & \text{otherwise}
    \end{cases}
$$
$$
    \left[ \pmb{B}_{n \times k} \right]_{i,j}
    = \begin{cases}
        1 & \text{if $s_j \in L_i$} \\
        0 & \text{otherwise}
    \end{cases}
$$
$$
    \left[ \pmb{C}_{n \times k} \right]_{i,j}
    = \begin{cases}
        1 & \text{if $s_j \in N_i$} \\
        0 & \text{otherwise}
    \end{cases}
$$
$$
    \left[ \pmb{B} \pmb{A}^T \right]_{i,j}
    = \left[ \pmb{D}_{n \times m} \right]_{i,j}
    = \left| L_i \cap S_j \right|
$$
$$  \left[ \pmb{C} \pmb{A}^T \right]_{i,j}
    = \left[ \pmb{E}_{n \times m} \right]_{i,j}
    = \left| N_i \cap S_j \right|
$$

Zapis rozwiązania:
$$
    \left[ \pmb{R}_{n \times m} \right]_{i,j}
    = \begin{cases}
        1 & \text{if potrzebie $i$-tej została przydzielona pizza $j$-ta} \\
        0 & \text{otherwise}
    \end{cases}
$$
przy założeniu, że $\pmb{R} ~ \pmb{1}_{m \times 1} = \pmb{1}_{n \times 1}$.

Jakość dopasowania dla każdej potrzeby możemy wyliczyć następująco:
$$
    \pmb{d}_{n \times 1} = \left( \pmb{R} \odot \pmb{D} \right) \pmb{1}_{m \times 1}
$$
$$
    \pmb{e}_{n \times 1} = \left( \pmb{R} \odot \pmb{E} \right) \pmb{1}_{m \times 1}
$$
gdzie $\odot$ to iloczyn Hadamarda (element-wise multiplication).

Do obliczenia jakości rozwiązania możemy użyć funkcji:

Liczba dopasowań pozytywnych:
$$
    f(\pmb{R}) = \pmb{1}_{1 \times n} \pmb{d} = \pmb{1}_{1 \times n} \left( \pmb{R} \odot \pmb{D} \right) \pmb{1}_{m \times 1}
$$

Liczba dopasowań negatywnych:
$$
    g(\pmb{R}) = \pmb{1}_{1 \times n} \pmb{e} = \pmb{1}_{1 \times n} \left( \pmb{R} \odot \pmb{E} \right) \pmb{1}_{m \times 1}
$$

Liczba kawałków nietworzących całej pizzy:
$$
    h(\pmb{R}) = \left( \pmb{1}_{1 \times n} \pmb{R} \bmod p \right) \pmb{1}_{m \times 1}
$$

Funkcja celu:
$$
    \mathcal{F}(\pmb{R}) = \alpha f(\pmb{R}) + \beta g(\pmb{R}) + \gamma h(\pmb{R})
$$

Warunek:
$$
    \sum_{j = 1}^{m} c_j \left\lceil \frac{1}{p} \cdot \sum_{i = 1}^{n} \pmb{R}_{i,j} \right\rceil \leq c_{max}
$$
$$
    \left\lceil \frac{1}{p} \cdot \pmb{1}_{1 \times n} \pmb{R} \right\rceil \pmb{c} \leq c_{max}
$$

Przestrzeń rozwiązań spełniających powyższy warunek to rodzina macierzy:
$$
    \mathcal{R}_{c_{max}}^{n \times m}
    = \left\{
    \pmb{R}_{n \times m}
    ~\middle|~
    \pmb{R} ~ \pmb{1}_{m \times 1} = \pmb{1}_{n \times 1}
    \land
    \left\lceil \frac{1}{p} \cdot \pmb{1}_{1 \times n}  \pmb{R} \right\rceil \pmb{c}
    \leq c_{max}
    \right\}
$$

Szukane:
$$
    \pmb{R}^{*}
    = \argmin_{\pmb{R} \in \mathcal{R}_{c_{max}}^{n \times m}} \mathcal{F}(\pmb{R})
$$

\end{document}
